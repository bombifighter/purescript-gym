\documentclass[12pt]{article}
\title{cím}
\author{Mezei Botond, Szabó Benedek}
\date{2023}
\renewcommand{\contentsname}{Tartalomjegyzék}
\usepackage[utf8]{inputenc}
\usepackage{graphicx}
\graphicspath{ {./images/} }
\usepackage{wrapfig}
\renewcommand{\figurename}{Ábra}
\usepackage{amsmath}
\usepackage{subcaption}
\usepackage{amssymb}
\usepackage[export]{adjustbox}
\usepackage[magyar]{babel}
\usepackage[breaklinks]{hyperref}
\usepackage{microtype}
\usepackage[T1]{fontenc}
\usepackage{lmodern}
\usepackage{multirow}
\usepackage[table]{xcolor}
%\usepackage{natbib}

\setlength{\arrayrulewidth}{0.5mm}
\setlength{\tabcolsep}{7pt}
\renewcommand{\arraystretch}{2.5}

\textwidth=14cm \textheight=20cm
%\hoffset=-1cm
%\voffset=-1cm

\newtheorem{theorem}{Theorem}[section]
\newtheorem{acknowledgement}[theorem]{Acknowledgement}
\newtheorem{algorithm}[theorem]{Algorithm}
\newtheorem{axiom}[theorem]{Axiom}
\newtheorem{case}[theorem]{Case}
\newtheorem{claim}[theorem]{Claim}
\newtheorem{conclusion}[theorem]{Conclusion}
\newtheorem{condition}[theorem]{Condition}
\newtheorem{conjecture}[theorem]{Conjecture}
\newtheorem{corollary}[theorem]{Corollary}
\newtheorem{criterion}[theorem]{Criterion}
\newtheorem{definíció}[theorem]{Definíció}
\newtheorem{discussion}[theorem]{Discussion}
\newtheorem{example}[theorem]{Example}
\newtheorem{exercise}[theorem]{Exercise}
\newtheorem{explanation}[theorem]{Explanation}
\newtheorem{illustration}[theorem]{Illustration}
\newtheorem{lemma}[theorem]{Lemma}
\newtheorem{notation}[theorem]{Notation}
\newtheorem{problem}[theorem]{Problem}
\newtheorem{proposition}[theorem]{Proposition}
\newtheorem{remark}[theorem]{Remark}
\newtheorem{solution}[theorem]{Solution}
\newtheorem{summary}[theorem]{Summary}

\makeatletter
%\renewcommand*{\@biblabel}[1]{–}
\makeatother

\begin{document}
\begin{titlepage}

	\begin{center}
	\mbox{}\\
	\vspace{35mm}
	\huge
	\textsc{Diplomamunka}
	\end{center}

	
	\vspace{55mm}
	\Large
	\hspace*{\fill} 
	\textbf{Mezei Botond, Szabó Benedek}
	
	\begin{center}
	\vspace{55mm}
	Debrecen\\	
	2023
	
	\end{center}
\end{titlepage}
\begin{titlepage}

	\vspace*{-2cm}
	\hspace*{-1.5cm}

	\begin{center}
	\large
	Debreceni Egyetem
	
	Informatikai Kar	

	Számítógéptudományi Tanszék
	
	\vspace{17mm}
	\huge
	\LARGE
    \textbf{Üzleti folyamatok modellezése és ERP rendszer üzleti követelményspecifikáció készítése a Cívisagro Kft. számára}
	
	\vspace{9mm}
	\large
	\textsc{Diplomamunka}
	
	\normalsize
	\vspace{15mm}
	\textsc{Készítette:}
	
	\vspace{5mm}
	\Large
	\textbf{Mezei Botond és Szabó Benedek}
	
	\normalsize
	programtervező informatika szakos hallgatók
	
	\vspace{18mm}
	\textsc{Témavezető:}\\
	\vspace{5mm}
	\large
	Dr. Battyányi Péter \\
	\normalsize
	adjunktus\\
	
	\vspace{27mm}
	Debrecen\\	
	2023
	
	\end{center}
\end{titlepage}
\newpage
\tableofcontents
\newpage
\section*{Bevezetés}
\addcontentsline{toc}{section}{Bevezetés}

A huszonegyedik század harmadik évtizedében minden vállalatnak, cégnek vagy szervezetnek elengedhetetlenül szükségessé vált az informatika bevonása a mindennapi működésbe. A cégek versenyképességük megőrzése, a hatékonyság és profitabilitás növelése érdekében már gyakorlatilag egy mikrovállalkozás méretétől kezdve rászorulnak valamiféle speciális céges szoftver, vállalatirányítási rendszer használatára. Egy kicsit nagyobb vállalatnál pedig már egészen elképzelhetetlen lenne a működés e szoftverek nélkül. A költségcsökkentés és automatizálás mellett meg kell említeni az információ növekvő szerepét, egy vállalatirányítási rendszer biztosíthatja a gyors (valós idejű) információ hozzáférést a munkatársaknak, illetve adatokkal látja el a döntéshozókat.


\newpage
\section{Szakirodalmi áttekintés}

\subsection{Üzleti folyamatok modellezése, modellezési eszközök}
Ebben a fejezetben először ismertetem a modellezés motivációit, majd áttekintésre kerül néhány lehetséges modellezési eszköz, melyek üzleti folyamatok, workflow-k modellezése esetén szóba jöhetnek. \cite{AR}

\subsubsection{ARIS}

Az ARIS (Architecture of Integrated Information System – Integrált Információs Rendszerek Architektúrája) egy modellező eszközcsaládot jelent, mely számos lehetőséget kínál üzleti folyamatok tervezésére, elemzésére, dokujmentálására, implementálására és optimalizálására (\textit{Vidovic–Vuksic, 2003}). Vállalatok és egyéb intézmények belső működésének és külső kapcsolatainak leírására is alkalmas.

\begin{figure}
\includegraphics[width=\textwidth]{testimg}
\caption{Az ARIS objektumai, és az ARIS ház felépítése. A jobb oldali ábra forrása:  \textit{Fonó, 2005}}
\end{figure}

%******************példa képek:

%\begin{figure}
%\centering
%\includegraphics[width=0.7\textwidth]{org}
%\caption{Organogram.}
%\end{figure}

%\begin{wrapfigure}{r}{0.4\textwidth}
%\includegraphics[width=0.4\textwidth]{test1}
%\caption{Értékteremtő lánc diagram.}
%\end{wrapfigure}

%\begin{figure}[h]
%\centering
%\begin{subfigure}{0.45\textwidth}
%\includegraphics[width=8cm]{pelda1} 
%\end{subfigure}
%\begin{subfigure}{0.4\textwidth}
%\includegraphics[width=8cm]{pelda2}
%\end{subfigure}
%\caption{A háló $t_1$ tüzelése előtt és után}
%\end{figure}

%******************felsorolas
%\begin{definíció}
%A Petri háló ($PN$) egy ($P, T, E, w, m_0$) rendszer, ahol
%\begin{enumerate}
%\item $P$ a helyek véges halmaza,
%\item $T$ a tranzíciók véges halmaza feltéve, hogy $P \cap T = \emptyset$,
%\item $E \subseteq (P \times T) \cup (T \times P)$ az élek véges halmaza,
%\item $ w: E\rightarrow N^+$ a súlyfüggvény,
%\item $ m: P\rightarrow N$ a token-eloszlás függvény,
%\item $ m_0$ a kezdeti token eloszlás.
%\end{enumerate}
%A Petri háló struktúrát $(P, T, E, w)$ módon jelöljük. (\textit{Peterson., 1981})
%\end{definíció}

\newpage
\section{Anyag és módszer}

Ebben a fejezetbenaz elkészített anyagokban használt módszereket, a felhasznált technikákat ismertetem. A specifikációk elkészültekor Dr. Kusper Gábor és Dr. Radványi Tibor tananyagára támaszkodtam.


\newpage
\section{Eredmények és azok értékelése}

Ebben a dokumentumban a cég jelenlegi állapotát ismertettem. Megtalálható benne a cég alapvető jellemzése, főbb mutatói, szervezeti felépítése és a jelenleg használt megoldások.

\subsection{ Követelménylista}

A követelmények megírásakor a már meglévő többi dokumentumra alapoztam. Véleményem szerint a követelmények egyértelműek voltak az elkészítés pillanatában. Az igények sem változtak az alatt a kis idő alatt, amíg a dokumentumok elkészültek. A következő két táblázat néhány követelményt tartalmaz:

\newpage
\section{Következtetések és javaslatok}

A vizsgált és specifikációk készítésekor használt modellezési nyelvek (ARIS, BPMN, UML, Petri-hálók) közül én az ARIS-t találtam jelen feladathoz a legmegfelelőbbnek. A különböző diagramtípusok integrációja és azok rendszerezett kezelése mindenképp kiemeli a többi közül. Ezen felül leginkább a folyamatok leírásakor az általa kínált információtöbbletet is hasznosnak találtam, véleményem szerint a többinél nagyobb kifejezőerővel rendelkezik ez az eszköz. 

\newpage
\section*{Összefoglalás}
\addcontentsline{toc}{section}{Összefoglalás}

A mai korszerű vállalatoknak versenyképességük megőrzése céljából elengedhetetlenné vált üzleti szoftverek alkalmazása. Dolgozatom célkitűzése egy debreceni kereskedelmi profilú kisvállalkozás folyamatainak, működésének megismerése, modellezése és a számára készülő vállalatirányítási rendszer üzleti követelmény specifikációjának elkészítése volt. 


\addcontentsline{toc}{section}{Irodalomjegyzék}
%\setlength{\parindent}{0mm}

%\nocite{*}
%\bibliographystyle{agsm}

\newpage
\renewcommand{\refname}{Irodalomjegyzék}
\begin{thebibliography}{99}

\bibitem{AR} M. Araújo, L. Roque. 2009. Modeling Games with Petri Nets. DiGRA Conference, 2009 

\bibitem{BSH}
C. Brom, V. Šisler, and T. Holan. 2007. Story manager in ‘Europe 2045’ uses Petri nets. 
Proceedings of the International Conference on Virtual Storytelling (ICVS '07), vol. 4871 of 
Lecture Notes in Computer Science, pp. 38–50, Strasbourg, 2007, France.


\end{thebibliography}

\end{document}
